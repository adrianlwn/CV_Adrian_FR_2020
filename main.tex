\documentclass{article}
% Package Imports

\usepackage[a4paper, hmargin=1cm, vmargin=1.5cm]{geometry}
\usepackage[hidelinks]{hyperref}
\usepackage[absolute]{textpos}
\usepackage[UKenglish]{babel}
\usepackage[UKenglish]{isodate}
\usepackage[dvipsnames]{xcolor}
\usepackage{fontspec,xltxtra,xunicode}
\usepackage{titlesec}
\usepackage{fontawesome}

\pagestyle{empty}


% Color definitions
\definecolor{darkgray}{HTML}{555555} 
\definecolor{primary}{HTML}{000000} 
\definecolor{headings}{HTML}{6A6A6A}
\definecolor{subheadings}{HTML}{555555}

% Set main fonts
\defaultfontfeatures{Mapping=tex-text}
\setmainfont{Lato}[
    Path = fonts/,
    Extension = .ttf,
    UprightFont = *-Regular,
    BoldFont = *-Bold,
    ItalicFont= *-Italic
 ]


\newfontfamily{\headingfont}{Raleway}[
    Path = fonts/,
    Extension = .ttf,
    UprightFont = *-Regular,
    BoldFont = *-SemiBold
    ]

% Headings command
\titleformat{\section}{\color{headings}\headingfont\scshape\bfseries\LARGE}{}{0em}{}[{\titlerule[0.4pt]}]
\titlespacing{\section}{0pt}{3pt}{5pt}

% Subsection command
\titleformat{\subsection}[runin]{\color{primary}\headingfont\bfseries\large}{}{0em}{}[]
\titlespacing*{\subsection}{0pt}{0pt}{8pt}

% Description COmmand 
\newcommand{\descript}[1]{\color{darkgray}\normalfont \textbf{#1\\}}

\newcommand{\longdescript}[1]{\color{subheadings}\normalfont\small {#1\\} }

% Location command
\newcommand{\location}[1]{\color{primary}\headingfont {\hfill #1\\} }

% Section seperators command
\newcommand{\sectionsep}[0]{\vspace{-6pt}}
% Bullet Lists with fewer gaps command
\newenvironment{tightemize}{\vspace{-2\topsep}\begin{itemize}\itemsep1pt \parskip0pt \parsep0pt}{\end{itemize}\vspace{\topsep}}

% Name command
\newcommand{\namesection}[4]{
	\begin{center}
		\sffamily
		\headingfont\fontsize{35pt}{14pt}\selectfont\scshape #1 
			\headingfont\selectfont\scshape\bfseries #2
	\end{center}
	\vspace{-14pt}
		\begin{center} \color{subheadings}\normalfont\fontsize{11pt}{14pt}\selectfont #3
		\end{center}
	
	%\noindent\makebox[\linewidth]{\rule{\paperwidth}{0.4pt}}
	\vspace{-5pt}
	}

\newcommand{\mybullet}[1]{ \hspace{2pt}\textbullet{#1}\hspace{2pt} }
%%%%%%%%%%%%%%%%%%%%% BEGIN %%%%%%%%%%%%%%%%%%%%%%%%
%%%%%%%%%%%%%%%%%%%%% DOCUMENT %%%%%%%%%%%%%%%%%%%%%

\begin{document}
%%%%%%%%%%%%%%%%%%%% NAME %%%%%%%%%%%%%%%%%%%% 

\namesection{Adrian}{Löwenstein}{  \href{https://www.adrianlwn.com}{\faGlobe{} adrianlwn.com} | \href{https://github.com/adrianlwn}{\faGithubSquare{}  github}  | \href{https://www.linkedin.com/in/adrianloewenstein}{\faLinkedinSquare{}  linkedin} \\ \href{mailto:adrian.loewenstein@icloud.com}{\faEnvelope{} adrian.loewenstein@icloud.com} |  \href{tel:+33767885257}{\faPhoneSquare{} +33 7 67 88 52 57}  \\ Nationalité Française | Né le 21.11.1993 
 }

% ''Data Scientist doublement diplomé de l'EPFL et de Imperial College. \\ Focus sur le Machine Learning et les Énergies Renouvelables.'' \\\vspace*{8pt}



%%%%%%%%%%%%%%%%%%%% TECHNICAL SKILLS %%%%%%%%%%%%%%%%%%%% 

\section{Compétences Techniques}

\subsection{Certifications} 
\longdescript{\href{https://www.youracclaim.com/badges/b73a0c9c-6fdd-4d60-8df4-4297c0ebfe3f}{Azure Data Engineer Associate} [DP-200 + DP-201]
}
\sectionsep

\subsection{Programmation}
\longdescript{Python [Pytorch, Keras, Pandas, Spark, Scikit-Learn, Matplotlib, ... ] \mybullet{} C \mybullet{} C++ \mybullet{} Matlab \mybullet{} SQL \\ \hspace*{3pt} Pratiques Devops [Git, CI/CD] \mybullet{} Javascript [D3.js]}
\sectionsep

\subsection{Informatique} 
\longdescript{ Machine Learning Classique \mybullet{} Deep Learning [CNN, GAN, RNN, LSTM] \mybullet{} Natural Language Processing  \\ \hspace*{3pt} Computer Vision \mybullet{} Reinforcement Learning  \mybullet{} Inférence Probabiliste [Processus Gaussiens, Optimisation Bayésienne , VAE ]   } 
\sectionsep

\subsection{Génie Électrique}
\longdescript{  \mybullet{} Théorie du Contrôle \mybullet{} Model Predictive Control \mybullet{} Optimisation  \mybullet{} Marché de l'électricité \\  \hspace*{3pt}  Contrôle et modélisation du réseau électrique  \mybullet{} Stockage de l'énergie \mybullet{} Génération de l'énergie}
\sectionsep




%%%%%%%%%%%%%%%%%%%%% EXPERIENCE %%%%%%%%%%%%%%%%%%%%% 
\section{Expériences Professionnelles}

\subsection{Quantmetry}
\location{Paris, France | Sept 2019 - Today}
\descript{Junior Data Scientist}
\longdescript{Consultant Data Scientist pour différentes missions et entreprises : \\ \textbf{Computer Vision} : Développement, pour une startup de l'univers de la mode, de modèles de segmentation automatique d'images avec des techniques de \textbf{Computer Vision} et de \textbf{Deep Learning} avec Pytorch. Déploiement effectué via des APIs Django sur GCP. \\ \textbf{NLP} : Développement pour un acteur des médias médicaux, d'outils permettant le profilage et le clustering de médecins par l'analyses des contenus lus. Utilisation de techniques de NLP, de clustering et de réduction de dimension.\\ \textbf{Formation} : Création de supports et opportunité de donner des formations à des Data Scientists sur des sujets de NLP et de NLU.}
\sectionsep

\subsection{Ecole Polytechnique Fédérale de Lausanne | EPFL}
\location{Lausannne, Suisse | 2015 - 2018}
\descript{Assistant Enseignement}
%\longdescript{Electrical Systems and Electronics I | Dr. Adil Koukab | Spring 2017 \& 2018\\
%Electrotechnics I | Pr. Oliver Martin | Autumn 2015 \& 2016 \\
%Microcontrollers | Pr. Alexandre Schmidt | Spring 2015
%}
\longdescript{Assistant à l'enseignement pour plusieurs Professeurs lors de mes études à l'EPFL.}
\sectionsep
\subsection{Commissariat à L’Énergie Atomique | CEA}
\location{Le Bourget du Lac, France | Juin 2016 - Août 2016}
\descript{Stage de Master}
\longdescript{Stage à l'\textit{Institut National de l'Energie Solaire} (INES) - Développement et validation de de stratégies de management de l'énergie dans les Smart-Grids. Étude de modèles de batteries et implémentation dans un simulateur. Évaluation sur un Microgrid solaire de stratégies de charge de batteries.}
\sectionsep

%\subsection{Airbus Helicopters UK }
%\location{Oxford, Royaume-Uni |  Juin 2011 – Juillet 2011}
%\descript{Stage}
%\longdescript{Stage d'introcution à l'ingenieurie effectué entre le baccalauréat et le debut de mes études.}
%\sectionsep

%%%%%%%%%%%%%%%%%%%%% EDUCATION %%%%%%%%%%%%%%%%%%%%% 
\section{Éducation}
\subsection{Imperial College London}
\location{Londres, Royaume-Uni | Sept. 2019 }
\longdescript{MSc in Computing | \textbf{Machine Learning} | obtenu avec Mérite }
\sectionsep
\subsection{École Polytechnique Fédérale de Lausanne | EPFL }
\location{Lausanne, Suisse | Juil. 2018}
\longdescript{MSc en Génie Électrique  | \textbf{Energy \& Smart Grids Science} | Moyenne : 5.31 / 6.0}
\sectionsep

\subsection{Eidgenössische Technische Hochschule Zürich | ETHZ }
\location{Zürich, Suisse | Août 2015}
\longdescript{Échange pour la 3e année de BSc}
\sectionsep

\subsection{École Polytechnique Fédérale de Lausanne | EPFL}
\location{Lausanne, Suisse | Fév. 2016}
\longdescript{BSc en Génie Électrique | Moyenne : 4.8 / 6.0}
\sectionsep

%\subsection{Lycée Pasteur}
%\location{Neuilly-sur-Seine, France | July 2012}
%\longdescript{Preparatory Course for Engineering Schools}
%\sectionsep

%\subsection{Institut de la Tour}
%\location{Paris, France | July 2011}
%\descript{High-school | French Baccalaureate  - Highest Honors}
%\sectionsep

%%%%%%%%%%%%%%%%%%%% SOFT SKILLS %%%%%%%%%%%%%%%%%%%% 
\section{Compétences Soft}
\subsection{Qualités}
\longdescript{Raisonnement \mybullet{} Analytique \mybullet{} Adaptabilité \mybullet{} Curiosité \mybullet{} Créativité \mybullet{} Esprit d'équipe \mybullet{} Vulgarisation }
\sectionsep
\subsection{Langues}
\longdescript{Français [Maternelle] \mybullet{} Anglais [C1] \mybullet{} Allemand [B2] }
\sectionsep

%%%%%%%%%%%%%%%%%%%%% EXTRA CURICULAR %%%%%%%%%%%%%%%%%%%%% 
\section{Hors Curriculum}
\subsection{Associatif}
\longdescript{Association des étudiants en Génie Électrique de l'EPFL \mybullet{} Photographe pour le compte Instagram de l'EPFL [@epflstudents] }
\sectionsep
\subsection{Autres}
\longdescript{Photographie [Argentique] \mybullet{} Édition Vidéo \mybullet{} Randonnée à Vélo \mybullet{} Randonnée \mybullet{} Ski \mybullet{} Voyages [Pérou, Chine, Iran]} 
\sectionsep


%%%%%%%%%%%%%%%%%%%%% PROJECTS  %%%%%%%%%%%%%%%%%%%%% 

\section{Projets de Data Science}

\subsection{Gaussian Processes for Optimal Sensor Position}
\location{Imperial College, Londres | Été 2019}
\descript{Thèse de Master}
\longdescript{Utilisation des Processus Gaussiens pour calculer la position spatiale optimale de capteurs pour l'étude et la captation de donnée liées à la polution de l'air dans les grandes villes. Validation des résultats avec la Data Assimilation. Problème Big Data. Imperial College London \textbf{Data Science Institute}. 
}
\sectionsep



\subsection{NLP Challenge - SemEval 2019 Task 6}
\location{Imperial College, London | Spring 2019}
\descript{Codalab Competition | \href{https://github.com/adrianlwn/SemEval-2019-Task-6}{\faGithubSquare{} Github Repository}}
\longdescript{Classification de Tweets Offensifs. Obtention du  \textbf{5e meilleur resultat}. Utilisation de méthodes SOTA comme les GRU, LSTM, RNN or CNN. }
\sectionsep

\subsection{Tweet Awareness - Data Analysis}
\location{EPFL, Lausanne | Autumn 2017}
\descript{Projet de Groupe| \href{https://adrianlwn.github.io/Tweet-Awareness-Data-Story}{\faGlobe{} Data Story}}
\longdescript{Projet visant à mesurer l'attention portée par des evenements dramatiques autour du monde et comment cela se corrèle des des métriques de distance culturelles.  Extraction de données de twitter avec Python (Selenium, BeautifulSoup). Analyse de données avec Python (Pandas, Sklearn). Visulations des données avec Javascript (D3.js).
Projet de Group fait dans le contexte du cours de Pr. R. West.}
\sectionsep

%\subsection{Speech Recognition Challenge - Network Data Science}
%\location{EPFL, Lausanne | Autumn 2017}
%\descript{Kaggle Competition |  %\href{https://github.com/adrianlwn/NTDS-Speech-Recognition-Challenge
%}{\faGithubSquare{} Github Repository}}
%\longdescript{Network Tour of Data Science Project. Classifying noisy audio commands. Cleaning and cutting of Audio Signals. Feature Extraction using mel-cepstral cepstrum. Building of a graph and extraction the fiedler vectors from the Laplacian to cluster the graph and classify the audio signals.  }
%\sectionsep

\section{Projets de Smart Grid}

\subsection{Provision of Multiple Services to the Grid with Electrical-Vehicles}
\location{EPFL, Lausanne | Printemps 2018}
\descript{Thèse de Master |  \href{https://stisrv13.epfl.ch/masters/img/960.pdf}{\faBook{} Poster du Projet} }
\longdescript{Utilisation des Véhicules Électriques for fournir des service au réseau électrique, tel que le réglage en fréquence. Problème d'optimisation utilisant des données de déplacement en voiture pour déterminer la capacité en régulation for une commercialisation sur les marchés. Développé en Matlab, utilisant le solver Gurobi et YALMIP. Supervisé par Pr. C. Jones.}
\sectionsep

\subsection{Robust restoration in DG-incorporated distribution networks}
\location{EPFL, Lausanne | Automne 2017}
\descript{Projet de Semestre de Master }
\longdescript{Formulation et implémentation du problème de \textbf{Restauration} dans les réseaux électriques, un problème : Mixed-Integer-Non-Linear. Développé en Matlab, utilisant le solver Gurobi. Supervisé par Dr. R. Cherkaoui.}
\sectionsep

\subsection{ETR applied to Fault Detection in Power Networks}
\location{EPFL, Lausanne | Printemps 2017}
\descript{Projet de Semestre de Master | \href{https://infoscience.epfl.ch/record/256592?ln=en}{\faBook{} Conference Paper}}
\longdescript{Étude de l'application du principe de l'Electromagnetic Time Reversal (ETR), dans le contexte de la détection de défauts dans les réseaux électriques.  Supervisé par Pr. F. Rachidi.  Trois semaines passées à  l' \textbf{Amir-Kabir University à Téhéran, Iran} pour ce projet. }
\sectionsep

\subsection{H2O2 Fuel Cell and Electrolyser Analysis and Monitoring}
\location{EPFL, Lausanne | Printemps 2016 }
\descript{Projet de Bachelor | \href{https://desl-pwrs.epfl.ch}{\faGlobe{} EPFL Microgrid}}
\longdescript{Implémentations d'un système de \textbf{Monitoring} (en LabView) pour système de stockage d'électricité constitué d'une Pile à Combustible et d'un Électrolyseur dans le contexte du Microgrid de l'EPFL. Supervisé by Pr. M. Paolone.}
\sectionsep

\end{document}
